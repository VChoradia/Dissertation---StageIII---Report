\chapter{Similar Software Available for IoT Development}

A number of softwares are available in the market for free of cost which supports IoT development. Below are some of the most commonly-used software for such development. If in the later phases of the project, it is find that it is efficient to use these instead of native development, they will be adopted.

\begin{enumerate}
    \item \textbf{The Things Network (TTN)}\\
	Used: IoT cloud platform\\
TTN is an open-source, decentralized infrastructure dedicated to IoT. TTN uses LoRaWAN as its secure messaging protocol. Devices that support LoRa can connect to gateways scattered all around the world that bridge them with TTN. It enables community users to create applications for data viewing and interact with the service by sending data from other IoT devices to other registered devices and gateways.

\item \textbf{ThingsBoard}\\
	Used: Dashboard tool\\
ThingsBoard is an open-source IoT platform for data collection, processing, visualization, and device management. It enables device connectivity via industry standard IoT protocols - MQTT, CoAP and HTTP and supports both cloud and on-premises deployments. The benefits of using ThingsBoard are scalability, fault-tolerance and performance so that data is intact. \cite{39}

\item \textbf{ThingSpeak}\\
Used: Analysis and Visualization\\
For even more advanced functionalities, there is ThingSpeak which is an IoT analytics platform service. This service allows a user to aggregate, visualize and even analyse live data streams without having to write any code. What makes ThingSpeak special is that it supports MATLAB analytics functions; users can write and run MATLAB code for more advanced analyses. ThingSpeak can also integrate with TTN to retrieve data from connected devices. \cite{40}

\item \textbf{Adafruit IO}\\
Used: IoT cloud platform\\
Adafruit IO is a cloud service. It can display data online from sensors in real-time, and connect to devices and control components. With a dashboard creator feature built-in, the software is capable of handling and visualizing multiple feeds of data. Adafruit IO can also be integrated with applications such as IFTTT and Zapier and other web services such as Twitter, RSS feeds etc. \cite{41}
\end{enumerate}