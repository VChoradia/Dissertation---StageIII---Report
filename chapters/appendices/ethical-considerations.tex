\chapter{Ethical Considerations}
\label{app:ethical}

In deploying IoT (Internet of Things) devices for passive monitoring in healthcare, several ethical considerations must be taken into account to ensure responsible and ethical use of technology.
\begin{itemize}
    \item Ensuring data privacy and confidentiality is paramount, even in passive monitoring scenarios. Measures such as encryption, secure data transmission protocols, and access controls should be implemented to protect the confidentiality of collected data.
    \item While passive monitoring may not require active patient participation, it's essential to inform individuals about the presence and purpose of monitoring devices. Transparency about data collection activities helps build trust and respect for individuals' autonomy, even if explicit consent is not obtained.
    \item Clarifying data ownership and control is important, particularly regarding the rights and responsibilities of data stewards. Individuals should have clarity on who has access to their data and how it will be used, with mechanisms in place for individuals to access, modify, or delete their data as needed.
    \item In cases where algorithms are used to analyze collected data, transparency in algorithmic processes is crucial. Individuals should have visibility into how algorithms operate and make decisions to ensure accountability and mitigate risks of bias or discrimination.
    \item Consideration of the broader social implications of passive monitoring is essential. While these technologies offer potential benefits in terms of healthcare optimization, they may also raise concerns about surveillance and privacy infringement. 
\end{itemize}

\noindent For the entire development and testing stage, it is advisable to utilize solely the developer's own health data rather than actual human health data to ensure ethical and responsible practices.
