\chapter{Legal Considerations}
\label{app:legal}

\section{Data Protection Regulations}

The healthcare sector extensively employs information technologies, including wearable health monitoring devices and IoT (Internet of Things) devices, for real-time vital monitoring and data collection. These technologies involve the processing of significant amounts of personal data, necessitating adherence to data protection regulations. \\ \\
Within the European Union (EU), personal data processing in healthcare is regulated by the General Data Protection Regulation (GDPR), which came into effect on May 25, 2018. The GDPR mandates obligations for data controllers and processors, ensuring the protection of individuals' personal data. Developers and providers of healthcare IoT devices must formulate privacy policies that align with GDPR requirements to ensure compliance.

\section{Medical Regulations}

Regulatory oversight in the UK for medical devices and IoT devices in healthcare is managed by the Medicines and Healthcare products Regulatory Agency (MHRA). Medical devices, including IoT devices used for vital monitoring, are categorized based on risk levels, ranging from low risk (Class I) to high risk (Class III). Even low-risk devices are subject to regulatory requirements, such as documentation preparation, clinical evaluations, and post-production review procedures. IoT devices intended for vital monitoring purposes fall under the scope of medical device regulations. Manufacturers must adhere to MHRA guidelines, which may include obtaining certifications and conducting clinical evaluations. \\ \\
In compliance with regulatory guidelines, it is essential for developers and providers of IoT devices for vital monitoring to ensure transparency and accountability in data processing practices. Users should be informed about data collection processes and their rights regarding their personal data.

