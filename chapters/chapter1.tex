\chapter{Introduction}

\section{Motivation}
Healthcare professionals (HCPs) are exposed to various sources of stress and pressure in their daily work, especially those who work in high-risk environments such as operation theatres. These stressors can negatively affect their physical and mental health and well-being, as well as their performance and quality of care. Extensive research has been conducted on this topic indicating that HCPs are liable to experience a range of mental health issues, including depression \cite{ref1}, anxiety \cite{ref2}, burnout \cite{ref3}, and stress \cite{ref4}. A study which surveyed 3,700 public sector workers in the United Kingdom found that staff working for the National Health Service were the most stressed, with 61\% reporting feeling stress all or most of the time and as of 2015 59\% stating that stress has increased from previous years \cite{ref5}.\\

The well-being concerns for HCPs elevated when COVID-19 hit the world in 2019. A wealth of research was done around the world talking about anxiety, stress, depression and the psychological well-being of HCPs \cite{ref6}\cite{ref7}. This pressured healthcare institutions to work towards the health and well-being of their staff. More and more healthcare institutions worldwide are considering strategies for promoting a safety culture and staff well-being. This project aims to create an Internet of Things as a Service Platform (IoTaaS) which helps these institutions get real-time insights into how to support HCPs in stressful environments best.


% \section{Aims and Objectives}

% The primary aim of this project is to delve into the potential of non-intrusive wearable technologies for real-time health monitoring of healthcare professionals, particularly those working in high-stress environments such as operating theatres, emergency rooms, and intensive care units. The motivation behind this aim is the critical need to ensure the health and well-being of these professionals, as their physical and mental state can directly impact the quality of care they provide to patients. \\

% The secondary aim of this project (which could also be labelled as future work)  is to open doors to further research opportunities that will elevate our current understanding of stressors in high-stress healthcare environments, and explore potential interventions to mitigate their impact. To reach that stage, a lot of potential work can be done following the creation of the device:

%  \begin{enumerate}
%      \item Conduct a feasibility study to evaluate the usability and acceptability of the prototype device among a sample of healthcare professionals working in high-stress environments, using surveys and interviews.
%      \item To analyse the data collected by the prototype device and assess the impact of the device on the health and well-being of healthcare professionals. This could involve statistical analysis of health indicators and correlation studies to understand the relationship between health indicators and stress and fatigue levels.
%      \item Based on the findings from the feasibility study and data analysis, provide recommendations and guidelines for the integration of non-intrusive wearable technologies into the existing attire of healthcare professionals and the improvement of their health and well-being. This could also include suggestions for improving HCPs' health and well-being based on the data analysis's insights.
%  \end{enumerate}

% \noindent If the timeframe of the project allows, the project will further dig into the above-mentioned works.

\section{Aims and Objectives}

The central aim of this project is to design and implement VitalMonitor, an integrated healthcare management system utilizing non-intrusive wearable technology for the real-time health monitoring of healthcare professionals (HCPs). This endeavor targets professionals in high-stress environments, such as operating theatres and emergency rooms, where continuous health monitoring is vital for maintaining high-quality patient care.\\

\noindent The objective would be to develop a wearable IoT device that measures vital signs without interfering with the HCPs' tasks, ensuring minimal disruption to their workflow. Additionally, the project will integrate this device within an Internet of Things as a Service (IoTaaS) platform, creating a seamless interface where health data is accessible in real-time through a web dashboard tailored for each healthcare institution.\\

\noindent \noindent With VitalMonitor, the project aspires to cultivate a proactive healthcare culture that consistently safeguards HCPs' welfare, establishing a foundation for improved patient care.


\section{Overview of the Report}
The introductory chapter establishes the foundation of this project, elucidating the motivation behind the study. It emphasizes the increasing awareness of healthcare professionals’ health and well-being. \\

\noindent\textbf{Chapter 2}, delves into a literature survey, discussing the current strategies used by healthcare institutions and their inefficiencies. This chapter also discusses how IoT and IoTaaS came into the scenario and brought different wearable technologies which helped in digitalising healthcare but still has no solution that could help HCPs. \\

\noindent\textbf{Chapter 3},  navigates essential aspects of the project - essentially what will be developed. It provides an overview of how the system will work, what features will be implemented and success metrics. This chapter concludes with the risks identified. \\

\noindent\textbf{Chapter 4},  establishes the technical aspects of the project. Provides insights into what technologies will be needed for the success of the project. This chapter also includes a comparative study between various technologies for each component of the project and choosing the most suitable one for the project. \\

\noindent\textbf{Chapter 5}, gives a high-level overview of VitalMonitor and its components. It also discusses alternative 
design approaches to some of the requirements defined in Chapter 3. Chapters 3, 4 and 5 combined represents what could be seen as a brief Software and Hardware Requirements Specification (SHRS) document of the system, providing a foundational blueprint for its development and implementation. \\

\noindent\textbf{Chapter 6}, outlines the processes involved in the implementation and subsequent testing of the VitalMonitor system. It covers the development of firmware, middleware, and the user dashboard, employing an agile methodology to ensure adaptability and responsiveness throughout the project lifecycle. This chapter also describes various testing methodologies employed to validate the functionality and reliability of the system to ensure that all components meet the predefined specifications and work cohesively.\\

\noindent\textbf{Chapter 7},    discusses the final outcomes of the VitalMonitor system, highlighting the achievements and the impact of the solution. It critically analyzes the main limitations encountered during the project and suggests potential areas for future improvements. The evaluation of the system's performance and the feasibility of its application in real-world settings are also thoroughly examined.\\

\noindent\textbf{Chapter 8}, concludes the report by revisiting how developing this platform would help healthcare professionals and summarising why the proposed platform has shown potential to be widely used, if introduced to the market. The chapter reflects on the contributions of the project to the field of healthcare technology and outlines future research directions that could further enhance the system's functionality and marketability.
