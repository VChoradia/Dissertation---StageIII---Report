\newpage
\chapter{Literature Review}

This chapter critically explores the existing frameworks used by healthcare institutions to support the well-being of healthcare professionals (HCPs). It reviews the impact of stress and fatigue on HCPs within healthcare settings and evaluates how these challenges are currently being addressed. The chapter also highlights the significant role that emerging technologies, particularly the Internet of Things (IoT) and IoT-as-a-Service (IoTaaS), can play in enhancing the monitoring and management of healthcare environments. Additionally, it considers the shortcomings of current wearable technologies in providing non-intrusive, real-time health monitoring in remote settings. The review aims to underscore the necessity for innovative solutions that cater more effectively to the needs of HCPs. This would not only improve their well-being but the overall quality of patient care as well.

\section{Stress and Fatigue}

As per World Health Organization (WHO), Stress can be defined as a state of worry or mental tension caused by a difficult situation. Stress is a natural human response that prompts us to address challenges and threats in our lives. Everyone experiences stress to some degree. The way we respond to stress, however, makes a big difference to our overall well-being. Stress affects both the mind and the body. \cite{48} \\

\noindent Fatigue is a term used to describe an overall feeling of tiredness or lack of energy. It isn’t the same as simply feeling drowsy or sleepy. When you’re fatigued, you have no motivation and no energy. Being sleepy may be a symptom of fatigue, but it’s not the same thing. \cite{49}

\section{Stress and Fatigue in Healthcare Settings}
Stress in healthcare professionals often stems from the high-pressure environment of medical settings, where the stakes are exceptionally high. Common stressors include a) long hours, b) complex patient interactions, and c) the emotional burden of continuous care giving. This chronic stress can lead to psychological symptoms such as anxiety and depression, and physical symptoms like headaches and hypertension, ultimately impacting their work efficacy and personal wellbeing. \\ 

\noindent Fatigue among healthcare professionals is more than just feeling tired. It’s a profound exhaustion that can result from prolonged physical, mental, and emotional labor. It is exacerbated by factors such as irregular shift patterns, the physical demands of the job, and the mental load of making critical decisions. Chronic fatigue can impair cognitive functions, reduce job performance, and increase the likelihood of errors in patient care. \\

\noindent In healthcare settings, stress and fatigue are exacerbated by the need to provide constant, high-quality care amidst frequently understaffed conditions. The relentless pace and emotional intensity of healthcare work can lead to burnout, a state of emotional, physical, and mental exhaustion. This not only affects the health of healthcare professionals but also compromises patient safety and the overall effectiveness of healthcare delivery. This made healthcare instutions to take various strategies to improve HCPs wellbeing.


\section{Current Strategies of Healthcare Institutions and their Inefficiencies}

The healthcare sector has been proactive in adopting various strategies to enhance the well-being of healthcare professionals (HCPs), acknowledging the multifaceted challenges they face. Despite the theoretical benefits of these strategies, practical inefficiencies and limitations often undermine their effectiveness. Sections listed below  are different strategies which are used by healthcare instituions.

\subsection{Integrated Care Systems (ICSs)}

Integrated Care Systems (ICSs) have been a cornerstone in the UK's healthcare strategy. ICSs are designed to facilitate cohesive planning and service delivery across various healthcare domains, including both physical and mental health services, as well as social care. The overarching goal is to enable individuals to lead healthier and more independent lives through self-care and prevention \cite{ref9}. However, ICSs sometimes struggle with delivering integrated care effectively, leading to fragmented services that are poorly coordinated around HCPs' needs. This fragmentation not only detracts from the patient and HCP experience but also results in inefficient resource use, echoing the financial and resource inefficiencies previously identified in Clinical Commissioning Groups (CCGs) \cite{ref11,ref12}.

\subsection{Understanding Evolving Needs}

The \textbf{Understanding Evolving Needs} strategy, informed by Accenture's survey of 720 healthcare providers, seeks to align healthcare institutions with the shifting expectations of HCPs, a need that became particularly acute during the COVID-19 pandemic \cite{ref18}. While this adaptive approach is commendable, accurately identifying and addressing the evolving needs of HCPs presents significant challenges, with effectiveness varying based on numerous factors including the healthcare setting and the specific needs of HCPs \cite{ref13,ref14}.

\subsection{Workplace Interventions}

\textbf{Workplace Interventions}, such as stress management programs, mindfulness practices, and workload adjustments, have been implemented to improve well-being and mitigate burnout among HCPs \cite{ref10}. Nonetheless, their effectiveness can be inconsistent, largely dependent on how well these interventions are tailored and executed within the unique contexts of different healthcare settings \cite{ref15}.

\subsection{Questionnaires}

The use of \textbf{Questionnaires} to assess the health and well-being of HCPs reflects a direct approach to understanding their needs. However, this method often fails to capture real-time changes and may elicit responses that do not accurately reflect the HCPs' health status \cite{ref16}. HCPs might prioritize patient care over their well-being or misjudge their health status as balanced, leading to responses that may not be entirely accurate or reflective of their actual needs. \\

\noindent The inefficiencies and limitations inherent in these strategies underscore the pressing need for innovative solutions that offer more immediate and comprehensive support for HCPs. One promising direction is the adoption of \textbf{Wearable Technologies}, an Internet of Things (IoT) application in healthcare, which could provide real-time health monitoring and personalized support, addressing many of the current methodologies' shortcomings.

\section{Measuring Stress and Fatigue Levels in Real-time}
Currently, there is no universally accepted algorithm for precisely calculating stress and fatigue levels, as these conditions are influenced by a complex interplay of physical, psychological, and environmental factors. This lack of a standardized measurement approach makes it challenging to accurately assess and manage stress and fatigue, particularly in high-pressure environments like healthcare settings.

% https://www.medicalnewstoday.com/articles/stress-measurement#normal-stress-levels

\subsection{Current Practices}
Technological advancements have led to the development of devices that can measure indicators of stress in real-time. One notable example is the use of Electrodermal Activity (EDA) testing, popularized by devices like Fitbit. The EDA test measures the skin's electrical activity, which varies with the body’s sweat level— a physiological response associated with stress. Fitbit's implementation of this technology assesses stress levels by requiring users to engage in a two-minute meditation session, during which the device monitors changes in EDA. \cite{49} However, this method poses practical challenges for healthcare professionals (HCPs) who often cannot spare two uninterrupted minutes for such a test while on duty. Therefore, while valuable, this technology may not always be suitable for real-time stress monitoring in fast-paced healthcare environments.


\subsection{Next Best Alternative}
To adapt real-time stress and fatigue monitoring for healthcare settings, technologies need to utilize parameters that can be passively monitored without disrupting workflow. These parameters should be easily sensed via remote and non-intrusive devices, allowing HCPs to wear and use them continuously throughout their duties. By ``non-intrusive," it means the device should be lightweight, comfortable, and not interfere with their work. \\

\noindent Sensors which can detect such parameters should be integrated into devices that are easy to use and interpret. This ensures that HCPs can benefit from the insights provided without needing extensive training or causing disruption to their critical work. And one of the leading technologies which can help making such a device is Internet of Things.


\section{Internet of Things (IoT)}
The Internet of Things (IoT) describes the network of physical objects —``things”— that are embedded with sensors, software, and other technologies to connect and exchange data with other devices and systems over the Internet. These devices range from ordinary household objects to sophisticated industrial tools. With more than 7 billion connected IoT devices today, experts are expecting this number to grow to 22 billion by 2025. \cite{ref16}\\


\noindent IoT devices generate vast amounts of data - offering valuable insights for businesses and individuals. With advances in machine learning and analytics, coupled with IoT data, businesses can gather insights faster and more easily. The emergence of these allied technologies continues to push the boundaries of IoT and the data produced by IoT also feeds these technologies. \cite{ref16}

\section{IoT in Healthcare}
The Internet of Things (IoT) in healthcare, also known as the Internet of Medical Things (IoMT), refers to  it’s a network of medical devices, software, and tech solutions that can monitor patient health, manage treatments, and even assist in surgical procedures \cite{47}. These devices are equipped with internet connectivity, enabling them to communicate with healthcare systems and each other. This integration facilitates a broad spectrum of health services and medical applications. \\

IoMT encompasses a variety of devices from wearable health monitors to advanced surgical equipment. These tools collect, analyze, and transmit health data, providing real-time insights crucial for informed decision-making and tailored patient care. The use of IoMT leads to greater accuracy in diagnoses, improves treatment outcomes, and enhances the efficiency of healthcare operations.\\


% https://relevant.software/blog/iot-in-healthcare/#:~:text=IoT%20in%20healthcare%2C%20also%20known%20as%20IoMT%20solution,manage%20treatments%2C%20and%20even%20assist%20in%20surgical%20procedures.

\subsection{Wearables}

Wearables are IoT devices that can be worn or carried on the body. These devices can be separated into categories such as smartwatches, wristbands, and hearables. The personal data of the user can be monitored and measured through powerful microchips and smart sensors that are embedded within the wearable device. Connecting to additional devices using Bluetooth, WiFi or a cellular network can further enhance the user experience \cite{ref17}. They are the most used device for healthcare monitoring. 

\begin{itemize}
    \item \textbf{Smart Watch}: Smart watches stand out in the wearable technology landscape for their versatility and widespread adoption. As per Statista, 2023, smartwatches currently dominate the wearables market and have held a share of around 45 per cent since 2018. Most fitness trackers are built into smartwatches and can track steps, distance, heart rate, calories burnt, and other fitness metrics. They offer a comprehensive suite of features that go beyond fitness tracking to include notifications, apps, and sometimes even phone capabilities. Their appeal lies in their ability to provide a snapshot of an individual's health and daily activities directly from their wrist.
    \item \textbf{Chest Strap}: Chest straps are specialized wearables designed for precise heart rate monitoring. Unlike wrist-based devices, chest straps measure electrical signals directly from the chest, offering a higher level of accuracy. They are particularly favored by athletes and individuals engaged in rigorous physical activities for monitoring heart rate and performance metrics.
\end{itemize}

\subsection{Limitations of Wearables}
\begin{enumerate}
    \item \textbf{Remoteness} \\
    One significant limitation of current wearable technology is its remoteness or isolation from centralized healthcare systems. This disconnect hampers the potential for wearables to inform clinical decisions and provide a holistic view of a patient's health. \\
    While wearable devices are capable of collecting a broad spectrum of health data from healthcare professionals themselves, there remains a significant barrier in integrating this data with the centralized systems used in hospitals and clinics. This is because most of such devices are made with general consumers in mind and the manufacturers don't usually sell it to organization with the perspective of monitoring health of all the staff in an organization. This disconnect hinders the ability of these devices to inform the organization about the health and fatigue levels of healthcare professionals in real time, which could otherwise support better staffing decisions and reduce the risk of burnout and medical errors.

    \item \textbf{Non-Intrusiveness} \\
    In healthcare settings, the non-intrusiveness of wearables is crucial. However, even state-of-the-art smartwatches face restrictions in sensitive environments like operation theatres or Intensive Care Units (ICUs). According to NHS England's guidance on uniforms and workwear, wearing watches in clinical settings is discouraged due to the risk of harboring microorganisms and hindering effective hand hygiene \cite{ref20}. Moreover, wristwatches may interfere with medical procedures, posing risks during surgeries or delicate operations.
    
\end{enumerate}

\subsection{Non-Intrusive devices not used primarily in Healthcare}
Healthcare devices are subject to rigorous regulatory scrutiny to ensure they meet high standards of accuracy, reliability, and safety. The certification process for medical devices, which includes clinical trials and detailed documentation, is both time-consuming and expensive. This rigorous scrutiny is essential to prevent errors that could directly affect patient health but creates a significant barrier to entry for new and innovative non-intrusive technologies. Furthermore, different countries and regions have their own regulatory frameworks, which can complicate the process for manufacturers looking to market their devices globally.\\

\noindent While non-intrusive devices offer the convenience of passive monitoring, maintaining the accuracy and reliability of these devices remains a technological hurdle. Sensor technology, particularly in non-intrusive formats, must evolve to capture data with the same precision as more traditional, intrusive methods. For example, sensors that work under or through clothing or that can monitor biometrics without direct skin contact often face issues such as signal attenuation or interference. These limitations can result in less reliable data, which in turn may affect the clinical utility of the information these devices provide.\\


\noindent Therefore, no such device exists which is remote and non-intrusive which can be used for monitoring vitals of healthcare professionals while they are working.

\section{Internt of Things as a Service (IoTaaS)}
Internet of Things-as-a-Service is one of the latest iterations in the ``as-a-service" model that has become one of the popular technologies \cite{45}. IoTaaS, or IaaS, or Internet of Things as a Service, integrates IoT technology into a service model, allowing companies to use IoT solutions without managing the infrastructure. It became notably popular in the mid-2010s as businesses sought to enhance efficiency and data collection without the overhead of developing proprietary IoT systems. \\

\noindent IoTaaS has been a subject of extensive research and development, as evidenced by its frequent spotlight at the EAI International Conference on IoT as a Service and Consumer Electronics Showcase. The EAI IoTaaS 2024 conference, for instance, is set to highlight cutting-edge advancements in IoT technologies and their applications across various industries. These conferences serve as pivotal forums for exchanging ideas and fostering international collaboration among scholars and industry experts, thereby propelling the global IoT industry forward.\cite{46} 

\section{Existing IoTaaS Platforms in Healthcare}
IoTaaS has become a transformative force in healthcare, enabling sophisticated remote patient monitoring systems and real-time data transmission to medical professionals. Such capabilities have improved the management of chronic diseases, improving patient outcomes, and enhancing operational efficiencies within healthcare facilities. \\

\noindent Notable research presented at the EAI International Conference on IoTaaS 2017, such as the ``Home Healthcare Matching Service System Using the Internet of Things," illustrates the potential of IoTaaS to deliver medical care in more comfortable, patient-preferred settings outside of traditional hospital environments.\cite{44} This shift not only aligns with broader trends towards patient-centered healthcare but also highlights the adaptability and potential of IoTaaS to revolutionize medical care practices. \\

\noindent A lot of HealthTech Companies are making their mark by creating IoTaaS platforms for healthcare. Some of the major platforms are Philips HealthSuite, GE Healthcare's Edison Platform, and Cisco Healthcare IoT. These platforms integrate with state-of-the-art healthcare technologies to get patient's data and uses their home AI and data analytics skills to provide the data to the doctors. \\

\subsection{Comparison between Existing Platforms}

\noindent A comparison table is shown between some of the current industry-standard platforms in provided in Table \ref{tab:iot-comparison}. Some of these platforms call them Platform-as-a-Service because they only provide the virtual platform using pay-per-use model to the end organizations. In the table N/A means no information was available on the sources about that feature for that platform. 


\begin{table}[ht]
\centering
\begin{tabularx}{\textwidth}{|l|X|X|X|}
\hline
\textbf{Features} & \textbf{Philips HealthSuite} & \textbf{GE Healthcare's Edison Platform} & \textbf{Cisco Healthcare IoT} \\ \hline
\textbf{Secure Data Management} &  ✓ &  ✓ &  ✓ \\ \hline
\textbf{Provides advanced analytics and AI capabilities} &  ✓ &  ✓ &  ✓ \\ \hline
\textbf{Scalable} & ✓ & ✓ & N/A \\ \hline
\textbf{Cloud-based} &  ✓ &  ✓ & ✗ \\ \hline
\textbf{Multiple Devices can be Integration} & ✓ & ✓ & N/A \\ \hline
\textbf{Security} & N/A & N/A & ✓ \\ \hline
\textbf{Real-time data \& Contextualized Insights} & ✓ & ✓ & ✓ \\ \hline
\textbf{Focus on HCP Well-being use case} & ✗ & ✗ & ✗ \\ \hline
\end{tabularx}
\caption{Comparison of IoTaaS Platforms in Healthcare}
\label{tab:iot-comparison}
\end{table}

\subsection{Limitations of Existing Platforms}
As shown in the Table \ref{tab:iot-comparison}, all the platforms do not have a focus on HCP well-being. While these platforms employ advanced technologies to enhance healthcare management, they uniformly lack features specifically tailored to monitor the health and well-being of healthcare providers themselves. This oversight indicates a market opportunity for the development of a platform that addresses the needs of healthcare professionals, ensuring their well-being is also supported and monitored effectively.