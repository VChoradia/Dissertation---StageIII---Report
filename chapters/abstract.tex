\chapter*{\Large \center Abstract}

   % This research project explores the potential of non-intrusive wearable technologies in monitoring the health and well-being of healthcare professionals, specifically those working in high-stress environments such as operating theatres. The study is motivated by the critical need to ensure the health and safety of healthcare professionals, whose performance can directly impact patient outcomes. The project will investigate the feasibility of integrating wearable technologies, such as smart patches, into the existing attire of healthcare professionals. These devices, connected via the Internet of Things (IoT), could provide real-time monitoring of vital signs, enabling timely interventions if necessary. The research will address several challenges, including ensuring the accuracy and reliability of the devices, maintaining their non-intrusiveness, and addressing the well-being concerns of healthcare professionals. The project aims to contribute to the growing body of knowledge on wearable technologies in healthcare and highlight the potential of these technologies in improving patient care by ensuring the health and well-being of healthcare professionals. 

% This design and build project delves into the application of non-intrusive wearable technologies for monitoring the health and well-being of healthcare professionals, particularly in high-stress environments like operating theatres. It responds to the imperative of safeguarding the health and safety of these professionals, recognizing their direct influence on patient outcomes. The project centers on integrating wearable technologies seamlessly into healthcare attire. These devices, interconnected via the Internet of Things (IoT), offer real-time monitoring of vital signs, facilitating prompt interventions as needed. Emphasizing a proof-of-concept approach, the project will tackle challenges of ensuring device accuracy, reliability, and maintaining non-intrusiveness while prioritizing healthcare professionals' well-being. Significantly, it aims to develop a centralized solution for healthcare institutions, employing Smart Patch—an IoT device—to provide continuous real-time monitoring of vital signs for all healthcare professionals. This centralized system will display comprehensive data on a dashboard, enhancing situational awareness and enabling proactive healthcare interventions. This research aims not only to contribute to the expanding realm of wearable technologies in healthcare but also to underscore their potential in elevating patient care by ensuring the sustained health and well-being of healthcare professionals.


% This project introduces VitalMonitor, a tailored healthcare management and monitoring system for healthcare professionals through remote monitoring and non-intrusive design. By harnessing the power of Internet of Things (IoT) technology, VitalMonitor offers continuous, real-time monitoring of vital signs via Smart Patch - a wearable device. This system aims to elevate the vigilance and response capabilities of health institutions to their staff's well-being. The IoTaaS framework of VitalMonitor promises an all-in-one service, blending device and health data management smoothly; ensuring a seamless, integrated experience. With its core infrastructure, the project encapsulates a multifaceted approach—encouraging remote health supervision without imposing on the daily routines of healthcare professionals. The health data from the IoT device will feed directly into a web dashboard for each organization, presenting a clear, real-time view of staff health metrics. This centralized platform stands as a testament to the project's commitment to facilitating immediate and informed interventions, thereby preemptively addressing health concerns before they escalate. With VitalMonitor, the aim is to shift healthcare systems toward a more anticipatory and preventive stance, affirming the critical role of health monitoring in improving patient care by first ensuring the health of those who deliver it.  It underscores the indispensability of maintaining the health and well-being of healthcare professionals, ultimately reflecting an improved patient care environment.

% Project is on to make Internet of Things as a Service (IoTaaS) platform - VitalMonitor which is nothing but an IoT-driven Healthcare Management and Monitoring System for healthcare professionals (HCPs). When we talk about Healthcare it is always about the -patients and very less about HCPs. VitalMonitors offers continuous, real-time monitoring of vital signs via Smart Patch - a wearable IoT device. This system aims to elevate the vigilance and response capabilities of health institutions to their staff's well-being. 


ADD WORDS LIKE CLOUD-BASED, ENCAPSULATES, ABSTRACTION, IOTAAS AND IOT BOTH, 
This design and build project investigates the integration of non-intrusive wearable technologies—specifically Smart Patches—into the daily attire of healthcare professionals operating in high-stress environments, such as operating theatres. The initiative, named VitalMonitor, proposes an Internet of Things as a Service (IoTaaS) platform designed to provide continuous, real-time monitoring of vital signs. This technology aims to ensure the health and well-being of healthcare professionals, whose condition directly influences patient care outcomes. Challenges addressed in this study include ensuring the accuracy, reliability, and non-intrusiveness of the technology while maintaining user comfort and seamless integration into existing healthcare workflows. By developing a centralized dashboard for health data management, the project enables immediate interventions and proactive health monitoring, ultimately enhancing patient care by safeguarding the caregivers themselves. This research contributes to the body of knowledge on IoT applications in healthcare, emphasizing the critical role of wearable technologies in improving both healthcare delivery and professional well-being.

